% A5 Bible in 640 pages: 8 point
% ppdefault.tex
% ppenaf.tex
% pplayout.tex
%   parallel.tex
% default definitions that you might override per file
\def\ENpretitle{%
DIE BYBEL: NUWE TESTAMENT\par
HOLY BIBLE: NEW TESTAMENT}
\def\AFpretitle{%
DIE BYBEL: NUWE TESTAMENT\par
HOLY BIBLE: NEW TESTAMENT}
\def\ENtitle{King James Bible}

\def\ENsubtitle{kjvtex compact}
\def\ENmotto{The fruit of the righteous is a tree of life;\par and he that winneth souls is wise.}
\def\ENcopyright{You may copy freely and republish in whole, or in part.\par}
\def\ENpages{pages}%
\def\ENaspect{aspect}%
\def\ENlinespacing{line spacing}%
\def\mytexlanguage{english}%
\def\mysourcetext{english}%
\def\mypageadjustments{}% Any other page adjustments - crop package goes here
\def\mybooktitlepreamble{% before each new book title, we emit this:
    \needspace{\mybookspace}%
    \null%\par
}
\newcommand\mybooktitlename[1]{%
    \noindent
    \null
    \hfill\tiny #1\hfill\null
    \break\noindent\normalsize
}
\newcommand\mybooktitlebox[1]{%
    \noindent
    \null
    \hfill%center it
    \raisebox{0.6\baselineskip}[0.5\baselineskip][0pt]{%
        \mybooktitletextbf{\mybooktitlefont\mybooktitlesettings #1}}%
    \bookheadingstrut{2\baselineskip}% assume that the paragraph above did the trick ...
    \hfill%center it
    \null
}

\def\mybooktitlesettings{%
    \spaceskip=0.33em%  "Song of Solomon" title seems to use undersized spacing without this, unless you don't change spaceskip
    \relax\color{mygray}% Don't bleed through the page - just be lighter
}%
\def\mybooktitlesettingsz{\spaceskip=0.33em\relax}% For the first page
\newcommand\mybooktitletextbf[1]{\textbf{#1}}% bold in title
\newcommand\mysmallcaps[3]{#1\mysmallcapsfont{#2}}% bold in title
\def\mypreverse{}% set this to \par for verse-per-line
\def\mypreversei{}% set this to \par for verse-per-line
\def\mypreverseii{}% set this to \par for verse-per-line
\def\mypredocument{}% things we set before the document - like setting the language
\def\mydocumentsettings{%
  \rightskip=0pt plus 0.5em  % slightly ragged right - really improves readability, at the cost of 1.5% pages (12 pages over 800)
}
\def\myhypenpenalty{10}% We are really quite willing to hyphenate
%\def\myheadsep{0.0em}
\def\myverseboldness{\myboldness}% verse number font scaling
\def\myversefontstretch{\myfontstretch}% verse number font scaling
\def\mychapterfontstretch{\myfontstretch}% verse number font scaling
\def\mychapterfontname{\mysansfont}% verse number font scaling
\def\mychapterfontextra{\relax}% fontextra ... things like \color{mygray}
\def\myversescale{0.65}% verse number font scaling
\def\mychapterscale{0.90}% chapter number scale
%\def\mychapterspace{0.25\baselineskip}% needspace for chapter title (should be 2\baselineskip - but in multicolumn, it doesn't work)
\def\mychapterspace{0.45\baselineskip}% needspace for chapter title (should be 2\baselineskip - but in multicolumn, it doesn't work)
\def\mybookspace{6.0\baselineskip}% needspace for chapter title (should be 2\baselineskip - but in multicolumn, it doesn't work)
\def\mychapterraise{1pt}% push the chapter number up this much
\def\myverseraise{2pt}% verse number superscript scaling
\def\myversehspace{0.8pt}% space after verse number
\def\myverseahspace{\myversehspace}% space after verse number before A
\def\PPfontcustom{}
\def\myboldness{FakeBold=1.01,}\def\myboldness{}%FakeBold is binary on/off, it seems
%\def\myletterspace{LetterSpace=0 plus 4 minus 0}
\def\myextradisplay{}% extra formatting notes, e.g. if you squash kerning
\def\mylayoutblurb{No headings; verses share lines; ragged; psalm titles; epistle postscripts}
\def\mycolumnsep{1em}
\def\mycontentcolumns{2}
\def\mygrayshade{0.33}
\def\mylettrinefindent{3pt}
\def\mycolumnseprule{0em}
\def\PPnewparagraph{\par}% Paragraph break instruction
\def\PPbeforeoldtestament{\par\newpage}% NEW PAGE before old testament
\def\PPbeforenewtestament{\par\newpage}% NEW PAGE before new testament
\def\PPafternewtestament{}% Pages after new testament
\newcommand{\PPpsalmheading}[1]{{\em #1}\PPeparnoindent}% A psalm of psalmist
\newcommand{\PPsectionseparator}[1]{% END OF THE PROPHETS. or THE END.
    \PPonecolumn% we're actually still in the book
    \par\null\par% a little space
    {\hfill{\em #1}\hfill}\par% somehow this ends up right justified ... weird, but good
    }%
\newcommand{\PPepistleattribution}[1]{\par{\em #1}}% Written by dude from place
\newcommand{\hebrewprefix}[1]{\def\hebrewleader{\myhebrewfont #1\ }}%
\newcommand{\hebrewinfix}[1]{{\myhebrewfont #1}\kern 0.3em}%
% 
\newcommand\showbindingmarks[1]{%
\AddToHookNext{shipout/background}{%
    \put(#1,-0.11\paperheight){\rule{1pt}{1pt}}%
    \put(#1,-0.22\paperheight){\rule{1pt}{1pt}}%
    \put(#1,-0.33\paperheight){\rule{1pt}{1pt}}%
    \put(#1,-0.44\paperheight){\rule{1pt}{1pt}}%
    \put(#1,-0.55\paperheight){\rule{1pt}{1pt}}%
    \put(#1,-0.66\paperheight){\rule{1pt}{1pt}}%
    \put(#1,-0.77\paperheight){\rule{1pt}{1pt}}%
    \put(#1,-0.88\paperheight){\rule{1pt}{1pt}}%
}}
% \showchaptermark{0.#1}
\newcommand\showchaptermark[1]{%
\AddToHookNext{shipout/background}{%
    \put(\paperwidth-\oddsidemargin+2.5mm,#1\paperheight){\rule{\oddsidemargin-2.5mm}{2pt}}%
}}
\newcommand\oddbindingmarks{\showbindingmarks{0pt}}
\newcommand\evenbindingmarks{\showbindingmarks{\paperwidth-1pt}}
\def\myversefontname{\mysansfont}
\def\mysansfontii{\mysansfont}

% Pages before Genesis
\newcommand\myfrontmatter{%
    \mypagei%    blank
    \mypageii%   print info page
    \mypageiii%  title page
    \mypageiv%   blank again
}

\newcommand\myrearmatter{%
    % Nothing to see here, sorry - no maps, no concordance, no dictionary, no weights and measures
}
\newcommand\usepackages{}
\newcommand\PP[1]{#1}
\def\mytocwidth{0.90\linewidth}
\def\PPtocpre{}
\def\PPtocpost{}

%%%%%%%%%%%%%%%%%%%%%%%%%%
\newcommand\circlegreyoutline[1]{%
  \Circled[outer color=circlegrey,fill color=circlegrey,inner color=black,inner xsep=1pt,inner ysep=2pt]{#1}%
% fill color=black,inner color=white,inner xsep=0.2ex,inner ysep=0.5ex,#1]{\sffamily\:#2} \fi }
}
\newcommand\circlegreyfill[1]{%
    \raisebox{0.5pt}{%
        \Circled[outer color=circlegrey,fill color=circlegrey,inner color=black,inner xsep=1pt,inner ysep=2pt]{#1}
    }
% fill color=black,inner color=white,inner xsep=0.2ex,inner ysep=0.5ex,#1]{\sffamily\:#2} \fi }
}
% Code clone city:
\newcommand{\circlesharedverse}[2]{\raisebox{1.5pt}[0pt]{\circlegreyoutline{\myversefont\textbf{#1}}}\def\mythumbindex{#2}}%
\newcommand{\circleverse}[2]{\mypreverse\hebrewleader\def\hebrewleader{}\circlesharedverse{#1}{#2}}%

\newcommand\tmpverse{}
% Underline the next verse number, by redefining "verse"
\newcommand\circleoverrideverse{%
    \let\overse\verse\relax
    \renewcommand\verse[2]{%
        \let\verse\overse
        \circleverse{##1}{##2}%j\hskip-1pt
    }
}
% Add a top-of-page note to a verse
\newcommand\topnote[2]{%
    % \circleoverrideverse% decorate the verse number
    \boldverse
    \gdef\extramark{%
        \tinyfont%
        \spaceskip=\dimexpr\the\fontdimen2\font-\the\fontdimen4\font\relax% plus \the\fontdimen4\font % space between words: keep it constant, with tiny changes, no up-side
        #1\ #2\null
        \gdef\extramark{\null}%
    }%
}
% Make the verse bold by redefining the PP macro:
\newcommand{\boldverse}{\renewcommand{\PP}[1]{\textbf{##1}\renewcommand{\PP}[1]{####1}}}

%%%%%%%%%%%%%%%%%%%%%%%%%%%%%


 % default settings
\def\mycolumnsep{0.8em}
\def\mydocumentsettings{
%  \spaceskip=\dimexpr\the\fontdimen2\font-\the\fontdimen4\font\relax plus 0.07em % plus \the\fontdimen4\font % space between words: keep it constant, with tiny changes, no up-side
%  \rightskip=0pt plus 1.8em\relax% space on right hand margin: keep it small
%  \parfillskip=0pt plus 10em\relax%   last line of paragraph space: use up all of the spare space here.
%  \spaceskip=\dimexpr\the\fontdimen2\font-\the\fontdimen4\font\relax plus 0.8em % plus \the\fontdimen4\font % space between words: keep it constant, with tiny changes, no up-side
  %\spaceskip=0.4em plus 0.8em minus 0.2em\relax% plus \the\fontdimen4\font % space between words: keep it constant, with tiny changes, no up-side
  \spaceskip=0.25em plus 0.35em minus 0.05em\relax% plus \the\fontdimen4\font % space between words: keep it constant, with tiny changes, no up-side
  %\rightskip=0pt plus 1ex\relax% space on right hand margin: keep it small
  \rightskip=0pt\relax% space on right hand margin: keep it small
  \parfillskip=0pt plus \linewidth\relax%   last line of paragraph space: use up all of the spare space here.
  \emergencystretch=4em
}
\def\myhypenpenalty{10}%   Lower penalty means more hypenation and fewer pages
\def\showredgrid{%
    \usepackage{tikz}
    \usepackage{eso-pic}
    \AddToShipoutPicture{%
        \begin{tikzpicture}[overlay,remember picture]
        \draw[thick,red]
        (current page.north east)
        rectangle (current page.south west);
        \draw[red!30!white,thin]
        (current page.south west) grid[step=\baselineskip]
        (current page.north east);
        \end{tikzpicture}%
    }%
}%
\def\showcropmarksonpage{%
    \usepackage[
      % set width and height to a5 width and height + 6mm
      width=148truemm, height=210truemm,
      % use any combination of these options to add different cut markings
      % cam, axes, frame, cross, % frame is good for checking margins visually
      cam, 
      % set the type of TeX renderer you use
      pdftex,
      % center the contents of the page
      center
    ]{crop}
}%
\def\mypageadjustments{%
    %\showredgrid
    %FINAL PRINTING:
    \showcropmarksonpage
}% Any other page adjustments - crop package goes here
\def\mydisplayfontname{Graze Shorter}
\def\myfontname{[GrazeShorter.otf]}
\def\mysansfontiiname{Liberation Sans}
\def\mysansfontname{Liberation Sans}
%\def\mysansfontname{[GrazeShorter.otf]}
\def\myfontsize{9.50pt}
\def\mybaseline{9.50pt}
\def\myfontopts{
    BoldFont=[GrazeShorter-Bold.otf],
    ItalicFont=[GrazeShorter-Italic.otf],
    BoldItalicFont=[GrazeShorter-BoldItalic.otf],
}
%\def\myfontname{Noto Serif}
%def\myfontopts{}
%8.57*59/56 = 9.02910714285714285714
%0.968/59*56 = .91877966101694915240
% This works, but it seems neater to squash the font, rather than reduce the inter-character spacing:
\def\myfontstretch{0.925}
\def\usepackages{%
    \usepackage[tracking=alltext]{microtype}
    \usepackage{fp}% floating point arithmetic to calculate the actual stretch for Afrikaans
    \FPdiv{\myfontstretchAF}{\myfontstretch}{1.007109}%meh
    \FPround{\myfontstretchAF}{\myfontstretchAF}{3}
}
%\def\myletterspace{LetterSpace=0.0 plus 25.0}
%\def\myfontstretch{0.875}
\def\myletterspace{}
\def\myextradisplay{}
\def\mytoccolumns{3}
\def\mychapterscale{0.95}
\def\myversescale{0.80}
\def\myverseraise{1pt}% =~ (\baselineskip+\fontsize)/2 - \fontsize*\myversescale)
% Improvements post-publishing 2023
\def\mypredocument{%
    %\pretolerance=50
    \pretolerance=20
    \tolerance=800\relax  % tolerance for variation in spacing
    %\renewcommand{\headruleskip}{-2pt}
    \frenchspacing% Oh wow ... this saves two pages!
    \selectlanguage{english}
\hyphenation{Am-ra-phel}
\hyphenation{Ahi-me-lech}
\hyphenation{Bel-te-shaz-zar}
\hyphenation{Eli-phaz}
\hyphenation{Eli-shah}
\hyphenation{Nebu-chad-nez-zar}
\hyphenation{Padan-aram}
\hyphenation{Rama-thaim-zo-phim}
\hyphenation{Je-ho-sha-phat}
\hyphenation{Baal-sha-li-sha}
\hyphenation{Ma-ha-la-leel}
\hyphenation{Ar-ta-xer-xes}
\hyphenation{Sho-shan-nim}
\hyphenation{Jo-na-the-lem-recho-kim}
\hyphenation{Shu-shan-ed-uth}
\hyphenation{Nebu-chad-rez-zar}
\hyphenation{Reho-boam}
\hyphenation{sa-bach-tha-ni}
\hyphenation{Eph-pha-tha}
\hyphenation{Abel-beth-maachah}
\hyphenation{Abel-meholah}
\hyphenation{Abel-miz-raim}
\hyphenation{Abel-shittim}
\hyphenation{Abiez-rites}
\hyphenation{Adoni-bezek}
\hyphenation{Adoni-zedek}
\hyphenation{Adram-melech}
\hyphenation{Adra-myttium}
\hyphenation{Aphar-sachites}
\hyphenation{Aphar-sathchites}
\hyphenation{Aphar-sites}
\hyphenation{Aram-naha-raim}
\hyphenation{Arche-vites}
\hyphenation{Areo-pagite}
\hyphenation{Ashdoth-pisgah}
\hyphenation{Ata-roth-adar}
\hyphenation{Ata-roth-addar}
\hyphenation{Aznoth-tabor}
\hyphenation{Baal-athbeer}
\hyphenation{Baal-berith}
\hyphenation{Baal-hermon}
\hyphenation{Baal-perazim}
\hyphenation{Baal-shalisha}
\hyphenation{Baal-zephon}
\hyphenation{Bamoth-baal}
\hyphenation{Bar-tholomew}
\hyphenation{Bar-timaeus}
%\hyphenation{Bashanhavothjair}
%\hyphenation{Bathrabbim}
%\hyphenation{Beerlahairoi}
\hyphenation{Beero-thite}
\hyphenation{Beero-thites}
%\hyphenation{Beeshterah}
%\hyphenation{Before-time}
%\hyphenation{Belshazzar}
%\hyphenation{Belteshazzar}
%\hyphenation{Benejaakan}
%\hyphenation{Benjamites}
\hyphenation{Bero-dach-bala-dan}
\hyphenation{Beth-arabah}
\hyphenation{Beth-az-maveth}
\hyphenation{Beth-baal-meon}
\hyphenation{Beth-diblathaim}
\hyphenation{Beth-haccerem}
\hyphenation{Beth-hoglah}
\hyphenation{Beth-jeshi-moth}
\hyphenation{Beth-jesi-moth}
\hyphenation{Beth-lebaoth}
\hyphenation{Beth-lehemite}
\hyphenation{Beth-lehem-judah}
\hyphenation{Beth-maachah}
\hyphenation{Beth-marca-both}
\hyphenation{Beth-nimrah}
\hyphenation{Beth-pazzez}
\hyphenation{Beth-phelet}
\hyphenation{Beth-shemesh}
\hyphenation{Beth-shemite}
\hyphenation{Beth-shittah}
\hyphenation{Beth-tappuah}
\hyphenation{Biz-joth-jah}
%\hyphenation{Calebephratah}
\hyphenation{Canaan-ites}
\hyphenation{Canaan-itess}
\hyphenation{Canaan-itish}
\hyphenation{Caph-thorim}
\hyphenation{Caph-torims}
\hyphenation{Cappa-docia}
\hyphenation{Car-chemish}
%\hyphenation{Carmelitess}
\hyphenation{Chal-daeans}
\hyphenation{Chedor-laomer}
\hyphenation{Che-phar-haam-mo-nai}
\hyphenation{Chere-thims}
\hyphenation{Chere-thites}
\hyphenation{Chesul-loth}
\hyphenation{Chin-ne-reth}
\hyphenation{Chin-ne-roth}
\hyphenation{Chis-loth-tabor}
\hyphenation{Chu-shan-risha-thaim}
\hyphenation{Dabba-sheth}
%\hyphenation{Dalmanutha}
%\hyphenation{Damascenes}
%\hyphenation{Destruction}
%\hyphenation{Deuteronomy}
%\hyphenation{Diotrephes}
\hyphenation{Ebed-melech}
%\hyphenation{Ecclesiastes}
%\hyphenation{Elishaphat}
%\hyphenation{Elonbethhanan}
%\hyphenation{Epaphroditus}
%\hyphenation{Ephesdammim}
%\hyphenation{Ephraimite}
%\hyphenation{Ephraimites}
\hyphenation{Ephra-thite}
\hyphenation{Ephra-thites}
%\hyphenation{Epicureans}
%\hyphenation{Esarhaddon}
%\hyphenation{Eshkalonites}
%\hyphenation{Eshtaulites}
%\hyphenation{Ethiopians}
%\hyphenation{Euroclydon}
%\hyphenation{Evilmerodach}
%\hyphenation{Eziongaber}
%\hyphenation{Eziongeber}
%\hyphenation{Fearfulness}
%\hyphenation{Foolishness}
%\hyphenation{Fortunatus}
%\hyphenation{Foursquare}
%\hyphenation{Furthermore}
%\hyphenation{Galilaeans}
%\hyphenation{Gathhepher}
%\hyphenation{Gathrimmon}
\hyphenation{Gaza-thites}
\hyphenation{Gedera-thite}
\hyphenation{Gede-ro-thaim}
%\hyphenation{Gennesaret}
%\hyphenation{Gergesenes}
%\hyphenation{Gershonite}
%\hyphenation{Gershonites}
%\hyphenation{Geshurites}
%\hyphenation{Gethsemane}
\hyphenation{Gibea-thite}
%\hyphenation{Gibeonites}
%\hyphenation{Gileadites}
%\hyphenation{Girgashite}
%\hyphenation{Girgashites}
%\hyphenation{Gittahhepher}
%\hyphenation{Haahashtari}
\hyphenation{Habazi-niah}
\hyphenation{Hach-mon-ite}
\hyphenation{Hadad-rim-mon}
\hyphenation{Hamath-zobah}
\hyphenation{Hamme-datha}
\hyphenation{Hammo-leketh}
\hyphenation{Hammoth-dor}
\hyphenation{Hanoch-ites}
\hyphenation{Hashab-niah}
%\hyphenation{Hashbadana}
\hyphenation{Havoth-jair}
\hyphenation{Hazar-addar}
\hyphenation{Hazar-gaddah}
\hyphenation{Hazar-hatti-con}
\hyphenation{Hazar-maveth}
\hyphenation{Hazar-shual}
\hyphenation{Hazar-sus-ah}
\hyphenation{Hazar-sus-im}
\hyphenation{Haza-zon-ta-mar}
\hyphenation{Hazel-el-po-ni}
\hyphenation{Ha-ze-zon-tamar}
%\hyphenation{Hebronites}
\hyphenation{Hel-kath-haz-zurim}
\hyphenation{Hence-forth}
%\hyphenation{Hepherites}
\hyphenation{Hermo-genes}
\hyphenation{Hermon-ites}
\hyphenation{Hezron-ites}
\hyphenation{Hiera-polis}
\hyphenation{Hor-ha-gid-gad}
\hyphenation{Huph-am-ites}
\hyphenation{Husha-thite}
\hyphenation{Ishbi-benob}
\hyphenation{Ishbo-sheth}
\hyphenation{Ishmael-ite}
\hyphenation{Ishmael-ites}
\hyphenation{Ishmeel-ite}
\hyphenation{Ishmeel-ites}
\hyphenation{Israel-ites}
\hyphenation{Israel-itish}
\hyphenation{Ittah-kazin}
\hyphenation{Izehar-ites}
\hyphenation{Jaareo-regim}
\hyphenation{Jabesh-gilead}
\hyphenation{Jachin-ites}
\hyphenation{Jahleel-ites}
\hyphenation{Jahzeel-ites}
\hyphenation{Jashubi-lehem}
\hyphenation{Jashub-ites}
\hyphenation{Jebere-chiah}
\hyphenation{Jeezer-ites}
\hyphenation{Jegar-sahad-utha}
\hyphenation{Jeha-le-leel}
\hyphenation{Jehoia-chin}
\hyphenation{Jeho-nathan}
\hyphenation{Jeho-shabeath}
\hyphenation{Jeho-shaphat}
\hyphenation{Jehovah-jireh}
\hyphenation{Jehovah-nissi}
\hyphenation{Jehovah-shalom}
\hyphenation{Jerah-meel-ites}
\hyphenation{Jerub-be-sheth}
\hyphenation{Jesha-re-lah}
\hyphenation{Jesho-haiah}
\hyphenation{Jez-reelite}
\hyphenation{Jez-reelitess}
\hyphenation{Jiph-tha-hel}
\hyphenation{Jonathelem-rechokim}
\hyphenation{Josh-be-ka-shah}
\hyphenation{Jushab-hesed}
\hyphenation{Kadesh-barnea}
\hyphenation{Kadmon-ites}
\hyphenation{Kedesh-naphtali}
\hyphenation{Ke-he-la-thah}
\hyphenation{Keniz-zites}
\hyphenation{Keren-happuch}
\hyphenation{Kibroth-hat-taavah}
\hyphenation{Kir-ha-ra-seth}
\hyphenation{Kir-ha-re-seth}
\hyphenation{Kir-iath-aim}
\hyphenation{Kir-jath-aim}
\hyphenation{Kir-jath-arba}
\hyphenation{Kir-jath-arim}
\hyphenation{Kir-jath-baal}
\hyphenation{Kir-jath-huzoth}
\hyphenation{Kir-jath-jearim}
\hyphenation{Kir-jath-sannah}
\hyphenation{Kir-jath-sepher}
\hyphenation{Koha-thites}
\hyphenation{Kora-thites}
\hyphenation{Maacha-thite}
\hyphenation{Maacha-thites}
\hyphenation{Maaleha-crab-bim}
\hyphenation{Machir-ites}
\hyphenation{Mach-nade-bai}
\hyphenation{Magor-missa-bib}
\hyphenation{Maha-la-leel}
\hyphenation{Maha-neh-dan}
\hyphenation{Maher-shalal-hash-baz}
\hyphenation{Malchiel-ites}
\hyphenation{Malchi-shua}
\hyphenation{Manahe-thites}
%\hyphenation{Manassites}
%\hyphenation{Mattathias}
%\hyphenation{Mattithiah}
%\hyphenation{Mecherathite}
%\hyphenation{Mehetabeel}
%\hyphenation{Meholathite}
\hyphenation{Mel-chi-se-dec}
\hyphenation{Mel-chi-shua}
\hyphenation{Mel-chi-ze-dek}
\hyphenation{Mephi-bosheth}
\hyphenation{Merodach-bala-dan}
%\hyphenation{Meronothite}
\hyphenation{Meshele-miah}
\hyphenation{Mesheza-beel}
\hyphenation{Meshil-le-mith}
\hyphenation{Meshil-em-oth}
\hyphenation{Meshul-em-eth}
\hyphenation{Meso-pota-mia}
\hyphenation{Methe-gammah}
\hyphenation{Methu-selah}
\hyphenation{Mich-methah}
%\hyphenation{Midianites}
%\hyphenation{Midianitish}
\hyphenation{Mish-mannah}
\hyphenation{Mish-raites}
\hyphenation{Mis-rephoth-maim}
\hyphenation{Mith-redath}
\hyphenation{Moras-thite}
\hyphenation{More-sheth-gath}
%\hyphenation{Multitudes}
%\hyphenation{Muthlabben}
%\hyphenation{Naamathite}
\hyphenation{Nathan-melech}
%\hyphenation{Nebuchadnezzar}
%\hyphenation{Nebuchadrezzar}
%\hyphenation{Nebushasban}
\hyphenation{Nebu-zara-dan}
%\hyphenation{Nehelamite}
%\hyphenation{Nemuelites}
%\hyphenation{Nephishesim}
%\hyphenation{Nephthalim}
\hyphenation{Nergal-sharezer}
\hyphenation{Jesus}
%\hyphenation{Neto-phathi}
%\hyphenation{Neto-phath-ite}
%\hyphenation{Neto-phath-ites}
%\hyphenation{Nevertheless}
%\hyphenation{Nico-lai-tans}
%\hyphenation{Not-with-stand-ing}
%\hyphenation{Ones-ipho-rus}
%\hyphenation{Pahath-moab}
%\hyphenation{Par-shan-datha}
%\hyphenation{Pelethites}
%\hyphenation{Per-adven-ture}
%\hyphenation{Perez-uzzah}
%\hyphenation{Perizzites}
\hyphenation{Pharaoh-hophra}
\hyphenation{Pharaoh-necho}
\hyphenation{Pharaoh-nechoh}
%\hyphenation{Philadelphia}
%\hyphenation{Philippians}
%\hyphenation{Philistine}
%\hyphenation{Philistines}
%\hyphenation{Philologus}
%\hyphenation{Pihahiroth}
%\hyphenation{Pirathonite}
%\hyphenation{Potipherah}
%\hyphenation{Praetorium}
%\hyphenation{Presumptuous}
%\hyphenation{Professing}
\hyphenation{Rama-thaim-zophim}
\hyphenation{Ramath-lehi}
\hyphenation{Ramath-mizpeh}
\hyphenation{Ramoth-gilead}
%\hyphenation{Rechabites}
%\hyphenation{Recompense}
%\hyphenation{Regem-melech}
%\hyphenation{Remmonmethoar}
%\hyphenation{Reubenites}
%\hyphenation{Revelation}
%\hyphenation{Righteousness}
\hyphenation{Rimmon-parez}
\hyphenation{Romamti-ezer}
\hyphenation{Samari-tans}
\hyphenation{Samgar-nebo}
\hyphenation{Samo-thracia}
\hyphenation{Selaham-mahle-koth}
\hyphenation{Senna-cherib}
\hyphenation{Sephar-vaim}
\hyphenation{Sephar-vites}
\hyphenation{Shaal-abbin}
\hyphenation{Shaal-bonite}
\hyphenation{Shab-bethai}
%\hyphenation{Shahazimah}
%\hyphenation{Shallecheth}
%\hyphenation{Shalmaneser}
%\hyphenation{Shamsherai}
%\hyphenation{Shearjashub}
\hyphenation{Shecha-niah}
%\hyphenation{Shechemites}
%\hyphenation{Shelanites}
%\hyphenation{Shemidaites}
%\hyphenation{Shemiramoth}
\hyphenation{She-pha-thiah}
\hyphenation{She-pha-tiah}
\hyphenation{She-phu-phan}
\hyphenation{Shesh-baz-zar}
\hyphenation{Sheth-arboz-nai}
%\hyphenation{Shibboleth}
%\hyphenation{Shigionoth}
\hyphenation{Shi-hor-lib-nath}
\hyphenation{Shil-lemites}
\hyphenation{Shi-lon-ites}
\hyphenation{Shi-mea-thites}
\hyphenation{Shim-ron-ites}
\hyphenation{Shim-ron-meron}
\hyphenation{Shoshan-nim}
\hyphenation{Shoshan-ni-meduth}
%\hyphenation{Shuhamites}
\hyphenation{Shuma-thites}
\hyphenation{Shunam-mite}
\hyphenation{Shu-pham-ites}
\hyphenation{Shu-shan-ed-uth}
\hyphenation{Shuth-al-hites}
\hyphenation{Suc-coth-ben-oth}
\hyphenation{Su-cha-thites}
\hyphenation{Susan-chites}
%\hyphenation{Syriadamascus}
\hyphenation{Syria-maa-chah}
\hyphenation{Taanath-shiloh}
%\hyphenation{Tachmonite}
\hyphenation{Tahtim-hodshi}
\hyphenation{Tar-pel-ites}
\hyphenation{Te-haph-ne-hes}
\hyphenation{Tel-ha-re-sha}
\hyphenation{Theo-phi-lus}
\hyphenation{Thes-sa-lon-ians}
\hyphenation{Thes-sa-lon-ica}
\hyphenation{Thim-nath-ah}
\hyphenation{Ti-glath-pil-eser}
\hyphenation{Til-gath-pil-neser}
\hyphenation{Tim-nath-heres}
\hyphenation{Tim-nath-serah}
\hyphenation{Ti-ra-thites}
\hyphenation{Toba-doni-jah}
\hyphenation{Trogyl-lium}
\hyphenation{Under-standing}
\hyphenation{Uzzen-sherah}
\hyphenation{Uzziel-ites}
\hyphenation{Zam-zum-mims}
\hyphenation{Zaph-nath-paan-eah}
\hyphenation{Za-rea-thites}
\hyphenation{Za-reth-shahar}
\hyphenation{Zebu-lon-ite}
\hyphenation{Ze-bu-lun-ites}
\hyphenation{Ze-lo-phe-had}
\hyphenation{Zephon-ites}
\hyphenation{Ze-re-da-thah}
\hyphenation{Ze-rub-ba-bel}
\hyphenation{Zo-ra-thites}
\hyphenation{Zu-ri-shad-dai}
\selectlanguage{dutch}
\hyphenation{by-me-kaar-maak}
\selectlanguage{english}

}
% /end 

\newcommand\ppFrontMatter{%
    %\ppblankpage%   blank
    \pptitlepage%  title page
    \ppinfopage%   print info page
    %\ppblankagain%  blank again
}
\input{kjvtopics.tex}
\def\myrearmatter{
%    \ppblankpage% blank
    \ppgospeloutlinepage%
%    \ppinfopage%   print info page
%    \kjvwordlist
%    \kjvtopics
}
%To omit manuscript titles (and save a page!)
%%
% To omit binding marks
%\renewcommand{\oddbindingmarks}{}
%\renewcommand{\evenbindingmarks}{}

    
\def\biblISBN{Printed in South Africa by CTP Printers Cape Town\par
ISBN: 978-0-7961-6044-7\par
Layout \copyright Calvary Hope Baptist Church 2024, with this licence:\par
You may copy freely and republish in whole, or in part.\par}
% Job end of book doesn't play nice - it's just the same length as the page
%\def\PPendchapterjobxli{\null\par\null\par}% End of Job 41 - two blank lines
% Load default page counts, which is used for the print checker and thumbnail
\gdef \mypagecount {600}
\InputIfFileExists{\jobname.section}{}{}

%%%%%%%%%%%%%%%%%pplayout%%%%%%%%%%%%%%%%%%%%%%%%%%%%%%%%%%%%%%%%%%%%%%%%
%% !TeX document-id = {7879a0e9-c3ab-473f-99fc-635c7bcf5210}
% !TEX encoding = UTF-8 Unicode
% !TEX TS-program = XeLaTeX
\documentclass[8pt,twoside,openany,a4paper]{extbook}
%\usepackage[utf8]{inputenc}
% openany says chapters start on either page
% twoside says it's a real book
%\usepackage{microtype}% Improves spacing
%\usepackage{lastpage}
\usepackage{calc}
\usepackage{fontspec}
\usepackages
% \usepackage{xunicode}
% \usepackage[margin=0.8in]{geometry}
%\usepackage[left=1.5cm,top=2.0cm,right=1.5cm,bottom=1.5cm,bindingoffset=0.5cm]{geometry}
%\geometry{papersize={7.444in,9.681in},total={4.8in,6.8in}}%
% 235mm x 165mm -
%\usepackage[utf8]{inputenc}% XeLaTeX is utf-8 or nuts
%\usepackage[T1]{fontenc}% Don't do this - it causes pdf fonts not to happen .. supposedly makes french hyphenation a thing
\usepackage[maxfloats=512]{morefloats}
\setlength{\headsep}{3pt}% Seperation between text and body
\usepackage{lettrine}% Drop caps letters
\usepackage[bf,sf,center]{titlesec}% Required for modifying section titles -   bold, sans-serif, centered
% Grid sets rigid baseline and glue lengths to keep register-true - front and back pages aligned
%%%%%%%%%%%%%%%%%%%%%%%%%%%%%%%%%%%%%%%%%%%%%%%%%%%%%%%%%%%%%%%%%%%%%%%%%%%%%%%%%%%%%%%%%%%%%%%%%%%%%%%%%
%%%%%%%%%%%%%%%%%%%%%%%%%%%%%%%%%%%%%%%
%% NOTO SERIF EXTRACONDENSED:
%\def\myfontopts{Path=ExtraCondensed/,}
%\def\myfontname{NotoSerif-ExtraCondensed}
%\def\myfontsize{10.00pt}
%\def\mybaseline{10.50pt}
%\def\mysansfont{DejaVu Sans Condensed}
%\def\myfontstretch{1.00}\def\myboldness{FakeBold=1.00,}\def\myboldness{}
%%%%%%%%%%%%%%%%%%%%%%%%%%%%%%%%%%%%%%%
%%%%%%%%%%%%%%%%%%%%%%%%%%%%%%%%%%%%%%%%%%%%%%%%%%%%%%%%%%%%%%%%%%%%%%%%%%%%%%%%%%%%%%%%%%%%%%%%%%%%%%%%%
\newcounter{qmylines}
\usepackage[nocheck]{fancyhdr}% Headers on the page, left and right
\usepackage{needspace}% Don't start new books at the bottom of the page
\usepackage[english,dutch]{babel}% hypenation
\usepackage[fontsize=\myfontsize,baseline=\mybaseline]{grid}% Register-true printing (back and front of page match)
\usepackage[\mymargins]{geometry}%
\geometry{papersize={\mypapersize}}
% Bottom border is 0.7 instead of 1.0, because we're not using the footer, and it's hard to turn off
%\usepackage[left=0.7cm,top=1.0cm,right=0.7cm,bottom=0.9cm,bindingoffset=0.0cm,headheight=5mm]{geometry}%
%\usepackage[left=7mm,top=10mm,right=7mm,bottom=7mm,bindingoffset=8mm,headheight=5mm]{geometry}% Waterproof
%\usepackage[left=5mm,top=10mm,right=5mm,bottom=5mm,bindingoffset=5mm,headheight=5mm]{geometry}%
%\geometry{papersize={148mm,221mm}}% Waterproof
\mypageadjustments% Any other page adjustments - crop package goes here
%\usepackage[absolute,overlay]{textpos}% Put things in absolute positions (for thumb marker)
\usepackage[absolute]{textpos}% Put things in absolute positions (for thumb marker)
\TPGrid[0mm,10mm]{30}{38}% defines tiny boxes down right hand margin for book name - see "textblock" below.  For 66 books, we have 61 markers
%FIXME \setlength{\headsep}{\myheadsep}% Seperation between text and body
% 0.3528pt/mm ... 334pt,510pt
%\usepackage[grid]{mymulticol}
%%%%%%%%%%%%%%%%%%%%%%%%%%%%%%%%%%%%%%%%%%%%%%%%%%%%%%%%%%%%%%%%%%%%%%%%%%%%%%%%%%%%%%%%%%%%%%%%%%%%%%%%%
%%%%%%%%%%%%%%%%%%%%%%%%%%%%%%%%%%%%%%%%%%%%%%%%%%%%%%%%%%%%%%%%%%%%%%%%%%%%%%%%%%%%%%%%%%%%%%%%%%%%%%%%%
%\premulticols = 50pt
%\postmulticols= 20pt
%\multicolsep = 12pt plus 4pt minus 3pt
%
%This complains about initex being required ... 
%\usepackage{etoolbox}
%\usepackage{xltxtra}
%\usepackage{setspace}
\usepackage{paracol}
%%%%%%%%%%%%%%%%%%%%%%%%%%%%%%%%%
%\setulmarginsandblock{2cm}{2cm}{*}
%\setlrmarginsandblock{2cm}{2cm}{*}
%\checkandfixthelayout
%%%%%%%%%%%%%
%Nice spacing, but not grid-like:
%\setlength{\parindent}{0.5em}
\setlength{\parindent}{0.6em}
%\setlength{\parskip}{0cm plus4mm minus1pt}
\fancypagestyle{noheadings}{%
    \fancyhead[EL]{}% hush
    \fancyhead[ER]{}% hush
    \fancyhead[OL]{}% hush
    \fancyhead[C]{}% hush
    \fancyhead[OR]{}% hush
}
\fancypagestyle{sectionheadings}{%
    \fancyhead[EL]{\thepage}% hush
    \fancyhead[ER]{}% hush
    \fancyhead[OL]{}% hush
    \fancyhead[C]{\myheadingfont \textbf{\mycurrentsection}}% hush
    \fancyhead[OR]{\thepage}% hush
}
\fancypagestyle{endmatter}{%
    \fancyhead[EL]{\thepage}% hush
    \fancyhead[ER]{}% hush
    \fancyhead[OL]{}% hush
    \fancyhead[C]{\myendnotesfont \textbf{\leftmark}}% hush
    \fancyhead[OR]{\thepage}% hush
}
\def\thelastpage{}%
\gdef\mythumbindex{0}% Where to put the vertical thumb index
\pagestyle{fancy}
\fancyhf{}
% register true is way better ...
%\fancyhead[L]{\textsf{\leftmark}}% Top left header
%\fancyhead[EL]{ - \em\textsf{\rightmark} - }% Top left header
\fancyhead[ER]{\extramark\hspace{1em}}% This is the chapter marker block
\fancyhead[OL]{\extramark}% This is the chapter marker block
\newcommand\extramark{}
\fancyhead[OR]{\myheadingfont \textbf{\textemdash\ \leftmark\hspace{1em}\null}\PPthumbmarkright{\mythumbindex}\oddbindingmarks}% Top right header
\fancyhead[EL]{\myheadingfont \textbf{\textemdash\ \rightmark}\PPthumbmarkleft{\mythumbindex}\evenbindingmarks}% Top left header - leading dash is to discourage accidental reading of the heading
\fancyhead[C]{\myheadingfont {\thepage}}% Center footer - bold is less readable
\renewcommand{\headrulewidth}{0.0pt}% lines just amplify the skewness of skewly printed pages 
\renewcommand{\footrulewidth}{0.0pt}%
%\defaultfontfeatures{Scale=MatchLowercase}
\defaultfontfeatures{}
\gdef\myshowpageinfo{}
%%%%%%%%%%%%%%%%%%%%%%%%%%%%%%%%%%%%%
\fontspec{\myfontname}
\setmainfont[\myfontopts \myboldness FakeStretch=\myfontstretch,Scale=1.00]{\myfontname}% 1006 pages +2
\PPfontcustom
\newfontfamily\mynostretchfont[FakeStretch=1]{\mysansfont}% Used for page headings
\newfontfamily\myheadingfont[\myboldness FakeStretch=\myfontstretch,Scale=1.00]{\mysansfont}% Used for page headings
\newfontfamily\mysmallcapsfont[\myfontopts\myboldness FakeStretch=\myfontstretch,Scale=0.80]{\myfontname}% Used for (synthetic) smallcaps
\newfontfamily\mylatexfont[FakeStretch=1.0]{Liberation Serif}% For saying "Latex" without looking ugly
\newfontfamily\myhebrewfont[]{Liberation Serif}% For Psalm 119
%\fontspec{Gentium Book Plus}\setmainfont[Numbers=Uppercase]{Gentium Book Plus}% 947 pages +2
% \fontspec[Scale=0.8]{Times New Roman}% (TimesNewRoman1)
% map TimesNewRoman(0)/bx/sc (bold smallcaps) to Junicode(1)/m/sc (normal smallcaps of fake bold)
% \DeclareFontShape{EU1}{TimesNewRoman(0)}{m}{sc}{<->ssub * TimesNewRoman(1)/m/sc}{}

\newfontfamily\myversefont[\myboldness FakeStretch=\myversefontstretch,Scale=\myversescale]{\myversefontname}
\newfontfamily\mybooktitlefont[\myfontopts \myboldness FakeStretch=\myfontstretch,Scale=2.0]{\myfontname}
\newfontfamily\mychapterfont[\myboldness FakeStretch=\mychapterfontstretch,Scale=\mychapterscale]{\mychapterfontname}
\newfontfamily\myendnotesfont[\myboldness FakeStretch=\myfontstretch,Scale=\mychapterscale]{\mysansfontii}
\usepackage{printlen}% provide \printlength command, and \rndprintlength - to print the page size nicely
\usepackage{paracol}
\usepackage{xcolor}

\definecolor{mygray}{gray}{\mygrayshade}
\setlength{\columnsep}{\mycolumnsep}
\setlength{\columnseprule}{\mycolumnseprule}%
%\setlength{\premulticols}{0\baselineskip}
%\setlength{\postmulticols}{0\baselineskip}
%\setlength{\multicolsep}{0\baselineskip}
\setlength{\emergencystretch}{0\baselineskip}% we don't care what your emergency is: don't stretch
%
%
%
% \pretolerance is used to tell the processor when there is too much space between words, and an attempt should be made to hyphenate some of them; 
% \tolerance is used to indicate when there is too much space even after breaking words.
% If we want to avoid breaking words, and avoid having too long lines, we have to use a very high value for 
% \pretoletance. An “infinite” tolerance is represented by the value 10000.
\pretolerance=10
\let\textquotedbl="
\let\quotedblbase="
%
%% % due to a bug in memoir (as of 2015/8/24) [twoside] option will give a spurious space
%% % this will be fixed as memoir gets updated
%% % \counterwithin{section}{chapter}
%% % \counterwithin{chapter}{verse}
%% \chapterstyle{section}
%% \newcommand*{\thebooktitle}{}
%% \renewcommand*{\printbooktitle}[1]{\gdef\thebooktitle{#1}\booktitlefont #1}
%% \renewcommand*{\printchapternum}{\chapnumfot\thebooktitle\ \thechapter}
%% \renewcommand{\thesection}{\arabic{section}}
%% \newcommand{\marginbox}[1]{%
%%    \parbox[t][0pt]{0em}{\bfseries\huge\raggedleft\leavevmode #1}}
%%  \newcommand{\marginhead}[1]{%
%%    {\llap{\marginbox{#1}\kern1em}}}
%%  \setsecindent{0em}
%%  \setaftersecskip{0em}
%%  \setsecheadstyle{\marginhead}
\def\hebrewleader{}%
\newcommand{\sharedverse}[2]{\raisebox{\myverseraise}[0pt]{\myversefont\textbf{#1}}\def\mythumbindex{#2}}%3 points up, count height as 0 points, kern at end
\newcommand{\versehskip}{\hspace{\myversehspace}}
\newcommand{\versehskipa}{\hspace{\myverseahspace}}% Kern verses starting with capital A
\newcommand{\versei}[2]{\noindent\mypreversei \hebrewleader\def\hebrewleader{}}%3 points up, count height as 0 points, kern at end
\newcommand{\verseii}[2]{\mypreverseii\hebrewleader\def\hebrewleader{}\sharedverse{#1}{#2}}%3 points up, count height as 0 points, kern at end
\renewcommand{\verse}[2]{\mypreverse\hebrewleader\def\hebrewleader{}\sharedverse{#1}{#2}}%3 points up, count height as 0 points, kern at end
\newcommand{\ppspacing}{\addfontfeature{\myletterspace}}
\newcommand{\PPnew}{}
\newcommand{\PPnewchapter}{}
\newcommand{\PPendchapterAF}{}
\newcommand{\PPendchapterEN}{}
\newcommand{\PPswlang}{}% Done with this language: fill up the column vertically
\newcommand{\PPnewlang}{}
\newcommand{\PPnewlangEN}{\begin{paracol}{2}\begin{leftcolumn}\selectlanguage{english}}
\newcommand{\PPnewlangAF}{\begin{rightcolumn}\selectlanguage{dutch}}
\newcommand{\PPnewbook}{\mybooktitlepreamble}
\newcommand{\PPbook}[1]{\mybooktitlebox{#1}}
\newcommand{\PPchapter}[1]{\PPdropcapschapter{#1}}
% Heading for each book of the PPe: big text
\newcommand{\PPverseref}[1]{\markboth{#1}{#1}}%
%\newcommand{\PPendreference}[2]{\markboth{#1}{#1}\def\mythumbindex{#2}}%
\newcommand{\PPversei}[3]{\PPverseref{#2}\versehskip\ppspacing#3}% the space after the verse is a thing
\newcommand{\PPverseii}[3]{\PPsyntheticparii\raisebox{\myverseraise}[0pt]{\myversefont\textbf{#1}}\versehskip\PPverseref{#2}\ppspacing#3}% the space after the verse is a thing
\newcommand{\PPverse}[3]{\par\raisebox{\myverseraise}[0pt]{\myversefont\textbf{#1}}\versehskip\PPverseref{#2}\ppspacing#3}% the space after the verse is a thing
\newcommand{\PPpostscript}[1]{} % omit postscripts - ignore parameter, which is the text of the postscript
\newcommand{\PPpar}{\PPsyntheticparii}
\newcommand{\PPpsalmauthor}[1]{{\ppspacing\em #1}\PPsyntheticparii}
\newcommand{\PPendlang}{}
\newcommand{\PPendlangEN}{\end{leftcolumn}}
\newcommand{\PPendlangAF}{\end{rightcolumn}\end{paracol}}
\newcommand{\PPendchapter}{}%
\newcommand{\PPendbook}{}
\newcommand{\PPend}{\par}
%\newcommand{\PPpagetitles}[2]{\lock{#1}{#2}}%
% FIXI?
\newcommand{\PPthumbmark}[3]{% 1:horizontal tab, 2: vertical tab, 3: hspace
\begin{textblock}{1}(#1,#2)%
%\rule[0em]{1em}{1ex}%
\hfuzz=72pt
\raisebox{0pt}[0pt]{\hspace{#3}\color{lightgray}\rule[0em]{8mm}{1.5em}}% vert offset, width, height
\end{textblock}
}
\newcommand{\PPthumbmarkright}[1]{\PPthumbmark{29}{#1}{0.5mm}}%
\newcommand{\PPthumbmarkleft}[1]{\PPthumbmark{0}{#1}{-4mm}}%
\newlength{\rightmarkspace}%
\newlength{\rightmarksep}
\setlength{\rightmarkspace}{\paperwidth-5em}% 
\setlength{\rightmarksep}{0.5ex}%
%\renewcommand{\PPtwocolumn}{\par}%
\newcommand{\PPenddocument}{}%
%
% \bookheadingstrut:
% * The 1.3\baselineskip is necessary sinnce the large text has a big baseline
% * 0pt makes it invisible
% * 4\baselineskip means it covers 4 lines
\newcommand{\bookheadingstrut}[1]{% %\vskip\baselineskip% disappearing space when at the top of the page? don't work :(
\rule[1.0\baselineskip]{0pt}{#1}% Number of lines for title - 1 down, two up
}%
% New chapter number 
% Move the thumb index (just make sure it doesn't disappear)
\newcommand{\PPtocendmulticol}{}%
\newcommand{\PPtocsectionheader}[1]{%
    \PPtocendmulticol%
    \begin{center}{\textbf{#1}}\end{center}%
    \begin{paracol}{\mytoccolumns}\sloppy%
    \renewcommand{\PPtocendmulticol}{\end{paracol}}%
}%
%
%\newcommand{\PPchapter}[2]{\vspace{2cm}\needspace{3\baselineskip}}
%% %New book in separate paracol
%% \begin{paracol}{2}%
%% \textbf{\Huge #1}%
%% \switchcolumn%
%% \textbf{\Huge #2}%
%% \end{paracol}\nopagebreak}%
%% \setlength{\abnormalparskip{1pt}}
%%%%%%%%%%%%%%%%%%%%%%%%%
%\newcommand{\PPdropcapschapter}[1]{\null\needspace{0.5\baselineskip}%
\newcommand{\PPdropcapschapter}[1]{\needspace{\mychapterspace}% big mystery here ... for needspace within 2 column environment, this is the number of lines /2 /2 (ie div 4)
%\lettrine[lines=2, lraise=0.05, lhang=0.00, findent=3pt, nindent=0pt, loversize=0.0]{\hspace{2pt}\textbf{\mychapterfont \smash{#1}}}{}}% bug: if you raise by 0.05, it breaks grid alignment for register-true 
\lettrine[lines=2, lraise=0.00, lhang=0.00, findent=\mylettrinefindent, nindent=0pt, loversize=0.0]{\raisebox{\mychapterraise}[0pt]{\textbf{\mychapterfont\mychapterfontextra\color{mygray} #1}}}{}%
}%
%debug: \textbf{#1}}%
% Work around lettrine bug with "Fpar" definition from https://tex.stackexchange.com/questions/369868/using-lettrine-with-short-paragraphs :
% Synthetic paragraph that should be used for verse 2, in case it is right net to verse 1 and the dropcaps chapter number
\def\PPeparnoindent{\hfil\vadjust{\vskip\parskip}\break}%
\def\PPsyntheticparii{\hfil\vadjust{\vskip\parskip}\break\indent}%
% Don't warn for slightly overfull hboxen
\hfuzz=2pt%
% All our vertical stuff is bad for now
\vbadness=10000%
\vfuzz=0pt%
% Try to avoid stranded chapter header at the bottom of the page (this doesn't work)
%\widowpenalty=0%
%\clubpenalty=0%
% Pretty separator
%\setlength{\columnseprule}{0.3pt}%
%\setlength{\headsep}{3pt}%
% Make leading dots for the table of contents - extra complex because text is too long sometimes:
\newcommand*{\mydotfill}[1]{%
    \noindent%
    \raggedright
    {\nobreak
    \hskip 0pt plus 1em minus 0pt
    \penalty50\null\nobreak
     \dotfill#1%
     \parfillskip=0pt \finalhyphendemerits=0 \par}
}
\mypredocument
\begin{document}
\hyphenpenalty=\myhypenpenalty
\mainmatter%
%FIXME - move this to \verse?
%\addfontfeature{\myletterspace}% The 3em space is what a good font design uses ... and we trample it
\mydocumentsettings
%%% PAGE 1: BLANK PAGE %%%%%%%%%%%%%%%
\def\mypagei{%
    \shipout\null\stepcounter{page}%this page intentionally left blank
}
%%% PAGE 2: BOOK INFO PAGE%%%%%%%%%%%%%%%
\def\myformattinginfoi{%
    \myshowpageinfo
    \ifdefined\paperwidth{%
        \uselengthunit{mm}\rndprintlength{\paperwidth}%
        $\times$%
        \rndprintlength{\paperheight}%
    }, \fi%
}
\def\myformattinginfoii{%
    \ifdefined\mydisplayfont%
        \mydisplayfont\ \myfontsize, %
    \else%
        \makeatletter\ifdefined\f@family\f@family\fi \makeatother\ \the\fontdimen6\font\the\font, %
    \fi
    \ENaspect\ \myfontstretch:1,
    \ENlinespacing\ \the\baselineskip
}
\def\myformattinginfoiii{%
    \ifdefined\myextradisplay{, \myextradisplay}\fi
}

\ifdefined\mypageii{}\else{
\gdef\mypageii{{
\spaceskip=0.33em\relax
\null\vfill
\InputIfFileExists{\jobname.pages}{}{}
\thispagestyle{noheadings}%
\ENpretitle\par
\ENtitle: \textsc{\ENsubtitle}
\par
\ifdefined\PPISBN{\PPISBN\par}\fi
\ENcopyright
\ifdefined\ENmottoii{\ENmottoii\par}\fi
\null\par
{\PPtocpre
Typesetting by {\mylatexfont \LaTeX{}}, %
\myformattinginfoi
\par
Font
\myformattinginfoii
\myformattinginfoiii%
\par
Source code: https://github.com/derde/kjvtex, ``\jobname.tex''
\par
Source text
\ifdefined\srcfilename{\srcfilename, }\fi
\ifdefined\srcmdsum{\srcmdsum\ MD5}\fi
\par
\newcounter{mylines}
\setcounter{mylines}{ \textheight / \baselineskip}
\newlength{\mybestlinespace}
\setlength{\mybestlinespace}{\textheight / \themylines}
Text area {\uselengthunit{mm}\rndprintlength{\textwidth}}$\times${\uselengthunit{mm}\rndprintlength{\textheight}} 
 (\arabic{mylines}$\times${\uselengthunit{pt}\rndprintlength{\mybestlinespace}})
\par
\mylayoutblurb
%\par\bookheadingstrut{3\baselineskip}%
\par
\PPtocpost
}
\newpage
}}}\fi
%%% PAGE 3: TITLE PAGE %%%%%%%%%%%%%%%
% Kerning adjustment: big deal page-count affecter
% Big deal don't hypenate rule: we would save pages by hypenating, but it's ugly
\def\mypageiii{
\thispagestyle{noheadings}%
%\fontsize{10}{10}\selectfont%
\null
%\bookheadingstrut{3\baselineskip}\par
%\vfill
\begin{center}%
\textsc{\mybooktitlesettingsz\ENpretitle}%HOLY BIBLE
\par
\bookheadingstrut{3\baselineskip}
\raisebox{0.6\baselineskip}[0.5\baselineskip][0pt]{{\mybooktitlefont\mybooktitlesettingsz\mybooktitletextbf{\ENtitle}}}% King James Bible
\par
\textsc{\mybooktitlesettingsz\ENsubtitle}% kjvtex compactt
\ifdefined\ENsubtitleii\ENsubtitleii\fi % soulwinning.info
\par
\vfill
\par
\ENmotto
\vfill
\end{center}
%%%%%%%%%%%%%%%%%%%%%%%%%%%
% emit table of contents, if we have it:
%% \newread\myfiletocread
%% \immediate\openin\myfiletocread={\jobname.section}
%% \loop\unless\ifeof\myfiletocread
%%     \read\myfiletocread to\fileline% Reads a line of the file into \fileline
%%     \fileline% just emit
%% \repeat
%% \closein\myfiletocread
\begin{center}
%\noindent\hspace{0.05\linewidth}% Indent the TOC so
\begin{minipage}{\mytocwidth}%
    \InputIfFileExists{\jobname.section}{}{}
\PPtocendmulticol
\end{minipage}
%%%%%%%%%%%%%%%%%%%%%%%%%%%
\vfill
\ifdefined\ENmottoii{\ENmottoii\par}\else
\myformattinginfoi \myformattinginfoii
\fi
\end{center}
%%%%%%%%%%%%%%%%%%%%%
% Commands that write the table of content later
%%%%%%%%%%%%%
\newpage
}
%%% PAGE 4: BLANK PAGE %%%%%%%%%%%%%%%
\def\mypageiv{%
    \shipout\null\stepcounter{page}%this page intentionally left blank
}
%%%%
\myfrontmatter
%\mypageiv%blank again
%%% PAGE 5: FIRST PAGE OF CONTENT %%%%%%%%%%%%%%%
\newcommand{\PPnewsection}[1]{
    \write\mytoc{\unexpanded{\PPtocsectionheader}{#1}}
    \gdef\mycurrentsection{#1}\thispagestyle{sectionheadings}
}%
\newwrite\mytoc\immediate\openout\mytoc=\jobname.section%
\pagestyle{fancy}
\include{\mysourcetext}% Load main content, from parallel.tex
%% Write out stats and stuff for laters
\newwrite\mypages
\immediate\openout\mypages=\jobname.pages%
\addtocounter{page}{-1}% adjust for presentation
\immediate\write\mypages{\unexpanded{\gdef\myshowpageinfo}{\thepage\ \ENpages\ }}% FIXME: adds one to the page .. weird
\addtocounter{page}{1}% un-adjust for presentation
% FIXME: this is dumb:
% I think we're done here ...
%% Oh wait, we might write out something more here
\myrearmatter% print the rear matter pages (not counted above!)
%%% PAGE n: RECORD PAGE COUNT AND QUIT %%%%
\addtocounter{page}{-1}% adjust for presentatoin
\immediate\write\mypages{\unexpanded{\gdef\mypagecount}{\thepage}}% FIXME: adds one to the page .. weird
\addtocounter{page}{1}% un-adjust for presentation
\immediate\closeout\mytoc%
\immediate\closeout\mypages%
\PPenddocument
%%\bigskip%%\ldots
\end{document}


% !TeX document-id = {7879a0e9-c3ab-473f-99fc-635c7bcf5210}
% !TEX encoding = UTF-8 Unicode
% !TEX TS-program = XeLaTeX
\documentclass[8pt,twoside,openany,a4paper]{extbook}
%\usepackage[utf8]{inputenc}
% openany says chapters start on either page
% twoside says it's a real book
%\usepackage{microtype}% Improves spacing
%\usepackage{lastpage}
\usepackage{calc}
\usepackage{fontspec}
\usepackages
% \usepackage{xunicode}
% \usepackage[margin=0.8in]{geometry}
%\usepackage[left=1.5cm,top=2.0cm,right=1.5cm,bottom=1.5cm,bindingoffset=0.5cm]{geometry}
%\geometry{papersize={7.444in,9.681in},total={4.8in,6.8in}}%
% 235mm x 165mm -
%\usepackage[utf8]{inputenc}% XeLaTeX is utf-8 or nuts
%\usepackage[T1]{fontenc}% Don't do this - it causes pdf fonts not to happen .. supposedly makes french hyphenation a thing
\usepackage[maxfloats=512]{morefloats}
\setlength{\headsep}{3pt}% Seperation between text and body
\usepackage{lettrine}% Drop caps letters
\usepackage[bf,sf,center]{titlesec}% Required for modifying section titles -   bold, sans-serif, centered
% Grid sets rigid baseline and glue lengths to keep register-true - front and back pages aligned
%%%%%%%%%%%%%%%%%%%%%%%%%%%%%%%%%%%%%%%%%%%%%%%%%%%%%%%%%%%%%%%%%%%%%%%%%%%%%%%%%%%%%%%%%%%%%%%%%%%%%%%%%
%%%%%%%%%%%%%%%%%%%%%%%%%%%%%%%%%%%%%%%
%% NOTO SERIF EXTRACONDENSED:
%\def\myfontopts{Path=ExtraCondensed/,}
%\def\myfontname{NotoSerif-ExtraCondensed}
%\def\myfontsize{10.00pt}
%\def\mybaseline{10.50pt}
%\def\mysansfontname{DejaVu Sans Condensed}
%\def\myfontstretch{1.00}\def\myboldness{FakeBold=1.00,}\def\myboldness{}
%%%%%%%%%%%%%%%%%%%%%%%%%%%%%%%%%%%%%%%
%%%%%%%%%%%%%%%%%%%%%%%%%%%%%%%%%%%%%%%%%%%%%%%%%%%%%%%%%%%%%%%%%%%%%%%%%%%%%%%%%%%%%%%%%%%%%%%%%%%%%%%%%
\newcounter{qmylines}
\usepackage[nocheck]{fancyhdr}% Headers on the page, left and right
\usepackage{needspace}% Don't start new books at the bottom of the page
%\usepackage[english,dutch]{babel}% hypenation
\usepackage{polyglossia}%
\setdefaultlanguage{english}
\setotherlanguage{afrikaans}
\setlanguagealias{afrikaans}{AF}
\setlanguagealias{english}{EN}
\usepackage{hyphenat}
\hyphenation{
% older forms missing from modern dictionary: *est
cove-nant-ed
cove-nant-break-ers Je-ru-sa-lem Oli-vet De-me-trius breth-ren Ti-mo-theus
ab-hor-rest ac-cep-test ap-prov-est ask-est bap-tiz-est bar-est bear-est be-hold-est
be-liev-est be-tray-est blas-phem-est bring-est build-est call-est cam-est
car-est chief-est com-est com-mand-est com-pel-lest con-demn-est con-si-der-est
could-est des-pis-est des-troy-est did-dest dis-hon-our-est do-est dwell-est earn-est
eld-est fast-est gav-est giv-est go-est great-est gu-est harv-est hat-est hear-est
high-est hold-est holi-est hon-est judg-est keep-est kill-est knew-est know-est
lack-est ledd-est l-est lett-est liv-est lov-est low-est mad-est mak-est
may-est might-est mod-est need-est ought-est ow-est par-tak-est per-ceiv-est
per-se-cut-est per-suad-est pray-est preach-est pri-est prot-est read-est reap-est
re-gard-est re-main-est re-mem-ber-est repli-est re-proach-est re-quest rest-est
sa-vour-est saw-est say-est seek-est se-est shew-est should-est sitt-est sleep-est
small-est smit-est sow-est speak-est spend-est stand-est ston-est strait-est
suf-fer-est tak-est talk-est tar-ri-est teach-est temp-est think-est tra-vail-est
trea-sur-est troubl-est visit-est walk-est weep-est went-est would-est wr-est
% older forms missing from modern dictionary: *eth
abid-eth abound-eth ac-cept-eth ac-cus-eth ack-now-ledg-eth add-eth ad-van-tag-eth
al-low-eth ans-wer-eth ap-pear-eth ap-proach-eth aris-eth as-cend-eth ask-eth
avail-eth bap-tiz-eth bear-eth beat-eth be-com-eth be-hav-eth be-hold-eth be-liev-eth
belong-eth be-tray-eth be-ray-eth bid-deth blas-phem-eth blow-eth boast-eth
bridl-eth bring-eth build-eth burn-eth buy-eth call-eth car-eth carri-eth cast-eth
catch-eth caus-eth ceas-eth chasten-eth cherish-eth cleans-eth cleav-eth climb-eth
com-eth comfort-eth com-mand-eth com-mend-eth com-mitt-eth con-demn-eth con-fess-eth
con-sist-eth con-strain-eth con-sult-eth con-tinu-eth con-vert-eth con-vinc-eth
cor-rupt-eth count-eth cover-eth cri-eth cum-ber-eth curs-eth deal-eth decay-eth
de-ceiv-eth de-fil-eth de-lay-eth de-ni-eth de-part-eth de-scend-eth de-scrib-eth
de-sir-eth de-spis-eth de-vour-eth dif-fer-eth dip-peth dis-an-nul-leth
dis-hon-our-eth di-vid-eth doubt-eth draw-eth drink-eth driv-eth dur-eth
dwell-eth eat-eth ed-ifi-eth en-dur-eth en-tangl-eth en-ter-eth envi-eth es-teem-eth
exalt-eth excell-eth exercis-eth exhort-eth fad-eth fail-eth fall-eth fear-eth
feed-eth fill-eth find-eth foam-eth fol-low-eth for-bid-deth for-get-teth
for-giv-eth for-sak-eth ga-ther-eth gen-der-eth giv-eth glori-eth gnash-eth 
greet-eth groan-eth grow-eth hang-eth harden-eth hat-eth hear-eth help-eth hid-eth
hold-eth hon-our-eth hop-eth humbl-eth im-put-eth in-creas-eth judg-eth jus-tifi-eth
keep-eth kill-eth kindl-eth knock-eth know-eth la-bour-eth lack-eth lay-eth
lead-eth lea-ven-eth leav-eth lett-eth light-en-eth light-eth limit-eth
linger-eth list-eth liv-eth lodg-eth look-eth los-eth lov-eth lust-eth mak-eth
marri-eth mean-eth mi-ni-ster-eth nam-eth need-eth nou-rish-eth oc-cupi-eth open-eth
op-pos-eth over-com-eth own-eth pass-eth pe-rish-eth pe-rsuad-eth per-tain-eth
per-vert-eth pin-eth plant-eth plow-eth pos-sess-eth pour-eth pray-eth preach-eth
press-eth pro-ceed-eth profit-eth pro-phesi-eth prosper-eth puff-eth purg-eth
pu-ri-fi-eth pur-pos-eth putt-eth quick-en-eth rais-eth read-eth reap-eth re-ceiv-eth
re-ckon-eth re-gard-eth reign-eth re-ject-eth re-joic-eth re-main-eth re-mem-ber-eth
re-pent-eth re-sist-eth rest-eth re-stor-eth ris-eth roar-eth rul-eth run-neth
salut-eth sanc-tifi-eth scat-ter-eth scourg-eth search-eth seek-eth seem-eth 
sell-eth send-eth serv-eth sett-eth sew-eth shew-eth shin-eth shoot-eth shut-teth
sig-ni-fi-eth sin-neth sitt-eth sleep-eth slum-ber-eth smi-teth sow-eth speak-eth
stab-lish-eth stand-eth step-peth stink-eth stir-reth strength-en-eth strik-eth
striv-eth stumbl-eth suf-fer-eth suf-fic-eth sup-pli-eth swear-eth tak-eth talk-eth
teach-eth tear-eth tell-eth tempt-eth tes-ti-fi-eth think-eth thresh-eth
touch-eth trans-gres-seth tra-vail-eth tread-eth tri-eth troubl-eth trust-eth
un-der-stand-eth up-braid-eth vanish-eth vaunt-eth wait-eth walk-eth warr-eth
watch-eth wa-ter-eth wa-ver-eth wax-eth wear-eth will-eth wi-ther-eth with-hold-eth
wit-ness-eth work-eth wor-ship-peth yield-eth Tro-phi-mus}
%%
\def\mymargins{left=10.0mm,top=10mm,right=7.0mm,bottom=7mm,bindingoffset=1mm,headheight=7mm}
\def\mypapersize{135mm,195mm}% A5, less 6.5mm right facing/reverse, and 7.5mm top and bottom
%%
\usepackage[fontsize=\myfontsize,baseline=\mybaseline]{grid}% Register-true printing (back and front of page match)
\usepackage[\mymargins]{geometry}%
\geometry{papersize={\mypapersize}}
% Bottom border is 0.7 instead of 1.0, because we're not using the footer, and it's hard to turn off
%\usepackage[left=0.7cm,top=1.0cm,right=0.7cm,bottom=0.9cm,bindingoffset=0.0cm,headheight=5mm]{geometry}%
%\usepackage[left=7mm,top=10mm,right=7mm,bottom=7mm,bindingoffset=8mm,headheight=5mm]{geometry}% Waterproof
%\usepackage[left=5mm,top=10mm,right=5mm,bottom=5mm,bindingoffset=5mm,headheight=5mm]{geometry}%
%\geometry{papersize={148mm,221mm}}% Waterproof
\mypageadjustments% Any other page adjustments - crop package goes here
\usepackage[absolute]{textpos}% Put things in absolute positions (for thumb marker)
%FIXME: Use \mypagecount+n from aux file:
\TPGrid[0mm,10mm]{30}{\mypagecount}% defines tiny boxes down right hand margin for book name - see "textblock" below.  For 66 books, we have 61 markers
%FIXME \setlength{\headsep}{\myheadsep}% Seperation between text and body
% 0.3528pt/mm ... 334pt,510pt
%\usepackage[grid]{mymulticol}
%%%%%%%%%%%%%%%%%%%%%%%%%%%%%%%%%%%%%%%%%%%%%%%%%%%%%%%%%%%%%%%%%%%%%%%%%%%%%%%%%%%%%%%%%%%%%%%%%%%%%%%%%
%%%%%%%%%%%%%%%%%%%%%%%%%%%%%%%%%%%%%%%%%%%%%%%%%%%%%%%%%%%%%%%%%%%%%%%%%%%%%%%%%%%%%%%%%%%%%%%%%%%%%%%%%
%\premulticols = 50pt
%\postmulticols= 20pt
%\multicolsep = 12pt plus 4pt minus 3pt
%
%This complains about initex being required ... 
%\usepackage{etoolbox}
%\usepackage{xltxtra}
%\usepackage{setspace}
\usepackage{paracol}
\ensurevspace{0pt}
%Save positions of verse references as we print
\usepackage[user,savepos,abspage]{zref}
%%%%%%%%%%%%%%%%%%%%%%%%%%%%%%%%%
%\setulmarginsandblock{2cm}{2cm}{*}
%\setlrmarginsandblock{2cm}{2cm}{*}
%\checkandfixthelayout
%%%%%%%%%%%%%
%Nice spacing, but not grid-like:
%\setlength{\parindent}{0.5em}
\setlength{\parindent}{0.6em}
%\setlength{\parskip}{0cm plus4mm minus1pt}
\InputIfFileExists{\jobname.headings.tex}{}{}
\fancypagestyle{noheadings}{%
    \fancyhead[EL]{}% hush
    \fancyhead[ER]{}% hush
    \fancyhead[OL]{}% hush
    \fancyhead[C]{}% hush
    \fancyhead[OR]{}% hush
}
\fancypagestyle{sectionheadings}{%
    \fancyhead[EL]{\thepage}% hush
    \fancyhead[ER]{}% hush
    \fancyhead[OL]{}% hush
    \fancyhead[C]{\myheadingfont \textbf{\mycurrentsection}}% hush
    \fancyhead[OR]{\thepage}% hush
}
\fancypagestyle{endmatter}{%
    \fancyhead[EL]{\thepage}% hush
    \fancyhead[ER]{}% hush
    \fancyhead[OL]{}% hush
    \fancyhead[C]{\myendnotesfont \textbf{\leftmark}}% hush
    \fancyhead[OR]{\thepage}% hush
}
\def\thelastpage{}%
\gdef\mythumbindex{0}% Where to put the vertical thumb index
\pagestyle{fancy}
\fancyhf{}
% register true is way better ...
%\fancyhead[L]{\textsf{\leftmark}}% Top left header
%\fancyhead[EL]{ - \em\textsf{\rightmark} - }% Top left header
%\fancyhead[ER]{\extramark\hspace{1em}}% This is the chapter marker block
%\fancyhead[OL]{\extramark}% This is the chapter marker block
\newcommand\extramark{}
\ifdefined\ppsave{\relax}\else{\newcommand{\ppget}[2]{\relax}}\fi
%\fancyhead[ER]{\myheadingfont \textbf{\textemdash\ \ppget{AFrange}{\thepage}\hspace{1em}\null}\PPbookmarker{\mythumbindex}\oddbindingmarks}% Top right header
\newcommand\pphcentre[1]{\makebox[0.46\linewidth]{#1}}
\newcounter{ppmarker}
\newcommand{\calcppmarker}{%  Marker at pagecount-currentpage
    \setcounter{ppmarker}{\mypagecount}%
    \addtocounter{ppmarker}{-\thepage}%
}
\fancyhead[ER]{\pphcentre{\myheadingfont \textbf{\ppget{AFrange}{\thepage}}}\calcppmarker\PPbookmarker{\theppmarker}}% Top right header
\fancyhead[OR]{\pphcentre{\myheadingfont \textbf{\ppget{AFrange}{\thepage}}}\PPpagemarker{\theppbookno}}% Top right header
\fancyhead[EL]{\pphcentre{\myheadingfont \textbf{\ppget{ENrange}{\thepage}}}}% Top left header - leading dash is to discourage accidental reading of the heading
\fancyhead[OL]{\pphcentre{\myheadingfont \textbf{\ppget{ENrange}{\thepage}}}}% Top left header - leading dash is to discourage accidental reading of the heading
\fancyhead[C]{\myheadingfont {\thepage}}% Center footer - bold is less readable
\renewcommand{\headrulewidth}{0.0pt}% lines just amplify the skewness of skewly printed pages 
\renewcommand{\footrulewidth}{0.0pt}%
%\defaultfontfeatures{Scale=MatchLowercase}
\defaultfontfeatures{}
\gdef\myshowpageinfo{}
%%%%%%%%%%%%%%%%%%%%%%%%%%%%%%%%%%%%%
\fontspec{\myfontname}
\setmainfont[\myfontopts \myboldness FakeStretch=\myfontstretch,Scale=1.00, \myletterspace]{\myfontname}% 1006 pages +2
\newfontfamily\myfontAF[\myfontopts \myboldness FakeStretch=\myfontstretchAF,Scale=1.0]{\myfontname}% Used for page headings
\newfontfamily\mynostretchfont[FakeStretch=1]{\mysansfontname}% Used for page headings
\newfontfamily\myheadingfont[\myboldness FakeStretch=\myfontstretch,Scale=1.00]{\mysansfontname}% Used for page headings
\newfontfamily\mysmallcapsfont[\myfontopts\myboldness FakeStretch=\myfontstretch,Scale=0.80]{\myfontname}% Used for (synthetic) smallcaps
\newfontfamily\mylatexfont[FakeStretch=1.0]{Liberation Serif}% For saying "Latex" without looking ugly
\newfontfamily\myhebrewfont[]{Liberation Serif}% For Psalm 119
%\fontspec{Gentium Book Plus}\setmainfont[Numbers=Uppercase]{Gentium Book Plus}% 947 pages +2
% \fontspec[Scale=0.8]{Times New Roman}% (TimesNewRoman1)
% map TimesNewRoman(0)/bx/sc (bold smallcaps) to Junicode(1)/m/sc (normal smallcaps of fake bold)
% \DeclareFontShape{EU1}{TimesNewRoman(0)}{m}{sc}{<->ssub * TimesNewRoman(1)/m/sc}{}

\newfontfamily\myversefont[\myboldness FakeStretch=\myversefontstretch,Scale=\myversescale]{\myversefontname}
\newfontfamily\mybooktitlefont[\myfontopts \myboldness FakeStretch=\myfontstretch,Scale=1.75]{\myfontname}
\newfontfamily\mychapterfont[\myboldness FakeStretch=\mychapterfontstretch,Scale=\mychapterscale]{\mychapterfontname}
\newfontfamily\myendnotesfont[\myboldness FakeStretch=\myfontstretch,Scale=\mychapterscale]{\mysansfontiiname}
\newfontfamily\myprefacefont[FakeStretch=\myfontstretch,Scale=0.9]{\mysansfontname}
\newfontfamily\mygospelfont[FakeStretch=\myfontstretch,Scale=0.75]{\mysansfontname}
\usepackage{printlen}% provide \printlength command, and \rndprintlength - to print the page size nicely
\usepackage{paracol}
\usepackage{xcolor}
\definecolor{mygray}{gray}{\mygrayshade}
%\colseprulecolor{mygray}%
%\setlength{\premulticols}{0\baselineskip}
%\setlength{\postmulticols}{0\baselineskip}
%\setlength{\multicolsep}{0\baselineskip}
%\setlength{\emergencystretch}{0\baselineskip}% we don't care what your emergency is: don't stretch
%
%
%
% \pretolerance is used to tell the processor when there is too much space
% between words, and an attempt should be made to hyphenate some of them;
% \tolerance is used to indicate when there is too much space even after
% breaking words.
% If we want to avoid breaking words, and avoid having too long lines, we have to use a very high value for 
% \pretoletance. An “infinite” tolerance is represented by the value 10000.
\let\textquotedbl="
%\let\quotedblbase="
%
%% % due to a bug in memoir (as of 2015/8/24) [twoside] option will give a spurious space
%% % this will be fixed as memoir gets updated
%% % \counterwithin{section}{chapter}
%% % \counterwithin{chapter}{verse}
%% \chapterstyle{section}
%% \newcommand*{\thebooktitle}{}
%% \renewcommand*{\printbooktitle}[1]{\gdef\thebooktitle{#1}\booktitlefont #1}
%% \renewcommand*{\printchapternum}{\chapnumfot\thebooktitle\ \thechapter}
%% \renewcommand{\thesection}{\arabic{section}}
%% \newcommand{\marginbox}[1]{%
%%    \parbox[t][0pt]{0em}{\bfseries\huge\raggedleft\leavevmode #1}}
%%  \newcommand{\marginhead}[1]{%
%%    {\llap{\marginbox{#1}\kern1em}}}
%%  \setsecindent{0em}
%%  \setaftersecskip{0em}
%%  \setsecheadstyle{\marginhead}
\def\hebrewleader{}%
%\newwrite\coords\immediate\openout\coords=\jobname.coords%
\newcommand{\ppsaveposition}[2]{% Save the print position, given a #1 label, and #2 reference
  \zlabel{z#1}%
  \zsavepos{#1}%
  %\edef\temp{#1,#2,x=\the\dimexpr\zposx{#1}sp\relax, y=\the\dimexpr\zposy{#1}sp,\zpageref{#1}}%
  %\immediate\write\coords{\temp,\thepage}%
}

\newwrite\myparaheight\immediate\openout\myparaheight=\jobname.paraheight%
\newwrite\myparapage\immediate\openout\myparapage=\jobname.parapage%
\newdimen\PPparaheight
\newcommand\PPmeasuretextEX[3]{\setbox0=\vbox{#1}%
    \PPparaheight=\ht0 \advance\PPparaheight by \dp0%
%Write to the ppenaf.heights file: but the page numbers here are NOT the page numbers for the headings
    \immediate\write\myparaheight{#2,#3,\the\PPparaheight}% Can't figure out how to merge these: merge them yourself
    \write\myparapage{\thepage}% pr -mt ppenaf.pagecontents ppenaf.heights
    #1%
}
%\newcommand\PPmeasuretext[1]{\PPmeasuretextEX{#1}{\somelanguage}{\someverse}}%
\newcommand\PPmeasuretext[1]{#1}% Non-measuring version
\newcommand{\sharedverse}[2]{\raisebox{\myverseraise}[0pt]{\myversefont\textbf{#1}}\def\mythumbindex{#2}}%3 points up, count height as 0 points, kern at end
\newcommand{\versehskip}{\hspace{\myversehspace}}
\newcommand{\versehskipa}{\hspace{\myverseahspace}}% Kern verses starting with capital A
\newcommand{\versei}[2]{\noindent\mypreversei \hebrewleader\def\hebrewleader{}}%3 points up, count height as 0 points, kern at end
\newcommand{\verseii}[2]{\mypreverseii\hebrewleader\def\hebrewleader{}\sharedverse{#1}{#2}}%3 points up, count height as 0 points, kern at end
\renewcommand{\verse}[2]{\mypreverse\hebrewleader\def\hebrewleader{}\sharedverse{#1}{#2}}%3 points up, count height as 0 points, kern at end
\newcommand{\ppspacing}{\addfontfeature{\myletterspace}}
\newcommand{\PPnew}{}
\newcommand{\PPnewchapter}{}
\newcommand{\PPendchapterAF}{}
\newcommand{\PPendchapterEN}{}
\newcommand{\PPswlang}{}% Done with this language: fill up the column vertically
\newcommand{\PPnewlang}{}
\newcommand{\pplang}[1]{}
\newcommand{\PPnewlangEN}{\begin{paracol}{2}\begin{leftcolumn}\renewcommand\pplang[1]{\begin{english}##1\end{english}}\gdef\somelanguage{en}}
\newcommand{\PPnewlangAF}{\begin{rightcolumn}\renewcommand\pplang[1]{\begin{afrikaans}\myfontAF##1\end{afrikaans}}\gdef\somelanguage{af}}
\newcounter{ppbookno}
\newcommand{\PPnewbook}{\mybooktitlepreamble\addtocounter{ppbookno}{18}}
%\newcommand{\PPbook}[1]{\mybooktitlebox{#1}}
\newcommand{\PPbook}[1]{}
\newcommand{\PPbooktitle}[2]{\mybooktitlename{#1}\mybooktitlebox{#2}}
\newcommand{\PPchapter}[1]{\PPdropcapschapter{#1}}
% Heading for each book of the PPe: big text
\newcommand{\PPverseref}[1]{\gdef\someverse{#1}\markboth{#1}{#1}}%
%\newcommand{\PPendreference}[2]{\markboth{#1}{#1}\def\mythumbindex{#2}}%
\newcommand{\PPversei}[4]{\PPmeasuretext{\PPverseref{#2}\ppsaveposition{#3}{#2}\versehskip\ppspacing\pplang{#4}}}% the space after the verse is a thing
\newcommand{\PPverseii}[4]{%
  %\ifdim\PPparaheight<1.1\baselineskip%
  %  \PPsyntheticpar% KQ% continue on same line
  %\else
    \PPsyntheticparii% QK%
  %\fi
\PPmeasuretext{\raisebox{\myverseraise}[0pt]{\myversefont\textbf{#1}}\ppsaveposition{#3}{#2}{\versehskip\PPverseref{#2}\ppspacing\pplang{#4}}}}% the space after the verse is a thing
\newcommand{\PPverse}[4]{\par\PPmeasuretext{\raisebox{\myverseraise}[0pt]{\myversefont\textbf{#1}}\ppsaveposition{#3}{#2}{\versehskip\PPverseref{#2}\ppspacing\pplang{#4}}}}% the space after the verse is a thing
\newcommand{\PPpostscript}[1]{} % omit postscripts - ignore parameter, which is the text of the postscript
\newcommand{\PPpar}{\PPsyntheticparii}
%\newcommand{\PPpsalmauthor}[1]{{\ppspacing\pplang{\em #1}}\PPsyntheticparii}
\newcommand{\PPhighlight}[1]{\marginpar{†}\textbf{#1}}%
\newcommand{\PPpsalmauthor}[1]{}% Omit
\newcommand{\PPendlang}{}
\newcommand{\PPendlangEN}{\end{leftcolumn}}
\newcommand{\PPendlangAF}{\end{rightcolumn}\end{paracol}}
\newcommand{\PPendchapter}{}%
\newcommand{\PPendbook}{}
\newcommand{\PPend}{\par}
%\newcommand{\PPpagetitles}[2]{\lock{#1}{#2}}%
% FIXI?
\newcommand{\PPthumbmark}[5]{% 1:horizontal tab, 2: vertical tab, 3: hspace, 4: lightgray
\begin{textblock}{1}(#1,#2)%
    %\rule[0em]{1em}{1ex}%
    \hfuzz=72pt
    \raisebox{0pt}[0pt]{\hspace{#3}\color{#4}\rule[0em]{8mm}{#5}}% vert offset, width, height
\end{textblock}
}
%\newcommand{\PPbookmarker}[1]{\PPthumbmark{29}{#1}{0.5mm}{lightgray}{1.5em}}%
%\newcommand{\PPpagemarker}[1]{\PPthumbmark{0}{#1}{-4mm}{lightgray}{0.5em}}%
\newcommand{\PPbookmarker}[1]{\PPthumbmark{0}{#1}{-5mm}{lightgray}{2mm}}%
\newcommand{\PPpagemarker}[1]{\PPthumbmark{29}{#1}{0.5mm}{lightgray}{7mm}}%
\newlength{\rightmarkspace}%
\newlength{\rightmarksep}
\setlength{\rightmarkspace}{\paperwidth-5em}% 
\setlength{\rightmarksep}{0.5ex}%
%\renewcommand{\PPtwocolumn}{\par}%
\newcommand{\PPenddocument}{}%
%
% \bookheadingstrut:
% * The 1.3\baselineskip is necessary sinnce the large text has a big baseline
% * 0pt makes it invisible
% * 4\baselineskip means it covers 4 lines
\newcommand{\bookheadingstrut}[1]{% %\vskip\baselineskip% disappearing space when at the top of the page? don't work :(
\rule[1.0\baselineskip-0.7pt]{0pt}{#1}% Number of lines for title - 1 down, two up ... the -0.7pt is because extra space is creeping in
}%
% New chapter number 
% Move the thumb index (just make sure it doesn't disappear)
% ppenaf.headings.tex defines PPtocAF and PPtocEN which format table-of-contents entries with PPtoc:
\newcommand\PPtoc[2]{\noindent\hspace{2em}#1 \mydotfill{ #2\hspace{2em}\null}}
%
%\newcommand{\PPchapter}[2]{\vspace{2cm}\needspace{3\baselineskip}}
%% %New book in separate paracol
%% \begin{paracol}{2}%
%% \textbf{\Huge #1}%
%% \switchcolumn%
%% \textbf{\Huge #2}%
%% \end{paracol}\nopagebreak}%
%% \setlength{\abnormalparskip{1pt}}
%%%%%%%%%%%%%%%%%%%%%%%%%
%\newcommand{\PPdropcapschapter}[1]{\null\needspace{0.5\baselineskip}%
\newcommand{\PPdropcapschapter}[1]{\needspace{\mychapterspace}% big mystery here ... for needspace within 2 column environment, this is the number of lines /2 /2 (ie div 4)
%\lettrine[lines=2, lraise=0.05, lhang=0.00, findent=3pt, nindent=0pt, loversize=0.0]{\hspace{2pt}\textbf{\mychapterfont \smash{#1}}}{}}% bug: if you raise by 0.05, it breaks grid alignment for register-true 
  \lettrine[lines=2, lraise=0.00, lhang=0.00, findent=\mylettrinefindent, nindent=0pt, loversize=0.0]{\raisebox{\mychapterraise}[0pt]{\textbf{\mychapterfont\mychapterfontextra\color{mygray} #1}}}{}%
}%
%debug: \textbf{#1}}%
% Work around lettrine bug with "Fpar" definition from https://tex.stackexchange.com/questions/369868/using-lettrine-with-short-paragraphs :
% Synthetic paragraph that should be used for verse 2, in case it is right net to verse 1 and the dropcaps chapter number
\def\PPsyntheticpar{\hfil\vadjust{\vskip\parskip}\break\noindent}%
\def\PPsyntheticparii{\hfil\vadjust{\vskip\parskip}\break\indent}%
% Don't warn for slightly overfull hboxen
\hfuzz=2pt%
% All our vertical stuff is bad for now
\vbadness=10000%
\vfuzz=0pt%
% Try to avoid stranded chapter header at the bottom of the page (this doesn't work)
%\widowpenalty=0%
%\clubpenalty=0%
% Pretty separator
%\setlength{\columnseprule}{0.3pt}%
%\setlength{\headsep}{3pt}%
% Make leading dots for the table of contents - extra complex because text is too long sometimes:
\newcommand*{\mydotfill}[1]{%
    \noindent%
    \raggedright
    {\nobreak
    \hskip 0pt plus 1em minus 0pt
    \penalty50\null\nobreak
     \dotfill#1%
     \parfillskip=0pt \finalhyphendemerits=0 \par}
}
\mypredocument
\begin{document}
\hyphenpenalty=\myhypenpenalty
\mainmatter%
%FIXME - move this to \verse?
%\addfontfeature{\myletterspace}% The 3em space is what a good font design uses ... and we trample it
\mydocumentsettings
%%% PAGE 1: BLANK PAGE %%%%%%%%%%%%%%%
\def\ppblankpage{%
    \shipout\null\stepcounter{page}%this page intentionally left blank
}
%%% PAGE 2: BOOK INFO PAGE%%%%%%%%%%%%%%%
\def\myformattinginfoi{%
    \myshowpageinfo
    \ifdefined\paperwidth{%
        \uselengthunit{mm}\rndprintlength{\paperwidth}%
        $\times$%
        \rndprintlength{\paperheight}%
    }, \fi%
}
\def\myformattinginfoii{%
    \ifdefined\mydisplayfontname%
        \mydisplayfontname\ \myfontsize, %
    \else%
        \makeatletter\ifdefined\f@family\f@family\fi \makeatother\ \the\fontdimen6\font\the\font, %
    \fi
    aspect\ \myfontstretch:1, (af) \myfontstretchAF:1,
    line spacing\ \the\baselineskip
}
\def\myformattinginfoiii{%
    \ifdefined\myextradisplay{, \myextradisplay}\fi
}
\def\ppPrefaceEN{
\myprefacefont
{\noindent\null\hfill \color{mygray}\textbf{Calvary Hope Baptist Church}\hfill\null}

\noindent This Bible has been printed by Calvary Hope Baptist Church (Cape
Town) for the furtherance of the Gospel of Jesus Christ in South Africa.  A
number of verses that are useful in teaching the Gospel of Christ with
simplicity have been highlighted in bold text.

The English and Afrikaans translations were translated from the Greek
Received Text: the English was informed by prior translations in English and
other languages, and the Afrikaans was based on the Dutch Statenvertaling.
Both the KJV and the 1953 Afrikaans are superior to the wholesale revisions
that followed them, because the revisions translate an inferior synthetic
modern revision of the text, and use faulty theories of translation.

The English text is the Bible of 1611 authorised by King James of England, with
the spelling and capitalisation revisions of 1769.  The Afrikaans text is the
Bible of 1933, published by the decision of hte join commission representing
the three Dutch Churches in South Africa, as revised in 1953 by the Bible
Society of South Africa.

}

\def\ppPrefaceAF{\begin{afrikaans}
\myprefacefont
{\noindent\null\hfill \color{mygray}\textbf{Calvary Hope Baptist Church}\hfill\null}

\noindent Dié Bybel is deur Calvary Hope Baptist Church (Kaapstad) gedruk vir die
bevordering van die Evangelie van Jesus Christus in Suid Afrika.  Enkele verse
wat dikwels gebruik word om die Evangelie van Chrisus met eenvoudigheid te leer
is in die teks beklemtoon.

Die Engelse en Afrikaanse vertalings is vertaal van die Griekse {\em Textus
Receptus}: die Engels was ingelig deur die vorige vertalings in Engels en ander
tale, en die Afrikaans het die Hollandse Statenvertaling as beginpunt gebruik.
Beide die Ou Vertaling van 1953 en die KJV is beter as die nuwe vertalings wat
op hulle gevolg het, omdat nuwe vertalings 'n minderwaardige kunsmatige moderne
Griekse teks vertaal, en foutiewe ideës oor vertaling gebruik.

Die Engelse teks is die Bybel van 1611 gemagtig deur Koning James van Engeland,
met die spelling en leesteken hersienings van 1769 (KJV).  Die Afrikaanse teks
is die Bybel van 1933 uitgesit in opdrag van die Gesamentlike Komissie
Vertenwoordigende die Drie Hollandse Kerke in Suid-Afrika, soos hersien in 1953
deur die Bybelgenootskap van Suid-Afrika. 

\end{afrikaans}
}

\gdef\ppinfopage{
% preface
\begin{paracol}{2}%
    \begin{leftcolumn}\ppPrefaceEN\end{leftcolumn}%
    \begin{rightcolumn}\ppPrefaceAF\end{rightcolumn}%
\end{paracol}
% skip
% \spaceskip=0.33em\relax
\null\vfill
\InputIfFileExists{\jobname.pages}{}{}
\thispagestyle{noheadings}%
DIE BYBEL: Nuwe Testament\par
HOLY BIBLE: New Testament\par
{\sc KJV 1953 Afrikaans Parallel Bible New Testament 2024}
\par
\ifdefined\PPISBN{\PPISBN\par}\fi
You may copy freely and republish in whole, or in part.\par
\ifdefined\ENmottoii{\ENmottoii\par}\fi
\null\par
{\PPtocpre
Typesetting by {\mylatexfont \LaTeX{}}, %
\myformattinginfoi
\par
Font
\myformattinginfoii
\myformattinginfoiii%
\par
Source code: https://github.com/derde/kjvpar, ``\jobname.tex''
\par
\input{md5sum.tex}
\PPmdsumEN\ \PPsourcefileEN\par
\PPmdsumAF\ \PPsourcefileAF\par
\par
\newcounter{mylines}
\setcounter{mylines}{ \textheight / \baselineskip}
\newlength{\mybestlinespace}
\setlength{\mybestlinespace}{\textheight / \themylines}
Text area {\uselengthunit{mm}\rndprintlength{\textwidth}}$\times${\uselengthunit{mm}\rndprintlength{\textheight}} 
 (\arabic{mylines}$\times${\uselengthunit{pt}\rndprintlength{\mybestlinespace}})
\par
Words of Christ in black.\par
%\par\bookheadingstrut{3\baselineskip}%
\par
\PPtocpost
}
\newpage
}


\gdef\ppGospelEN{%
    \begin{english}%
    \mygospelfont%
    \textbf{1. Every one is a sinner}

Romans 3:10: As it is written, There is none righteous, no, not one

Romans 3:23: For all have sinned, and come short of the glory of God


\textbf{2. Every one deserves to go to hell}

Romans 6:23: For the wages of sin is death; but the gift of God is eternal life through Jesus Christ our Lord

Revelation 20:14-15:  And death and hell were cast into the lake of fire. This is the second death.  And whosoever was not found written in the book of life was cast into the lake of fire

Revelation 21:8:  But the fearful, and unbelieving, and the abominable, and murderers, and whoremongers, and sorcerers, and idolaters, and all liars, shall have their part in the lake which burneth with fire and brimstone: which is the second death.


\textbf{3. Jesus died for our sins according to the scriptures; he was buried, and he rose again the third day according to the scriptures:}

Romans 5:8 But God commendeth his love toward us, in that, while we were yet sinners, Christ died for us.

1 Peter 2:24: Who his own self bare our sins in his own body on the tree, that we, being dead to sins, should live unto righteousness: by whose stripes ye were healed.


\textbf{4. You must believe on Jesus Christ as your only way to heaven}

Acts 16:31  And they said, Believe on the Lord Jesus Christ, and thou shalt be saved, and thy house.

John 3:16  For God so loved the world, that he gave his only begotten Son, that whosoever believeth in him should not perish, but have everlasting life.


\textbf{5. When you believe on Jesus Christ, God gives you eternal life (a gift illustrates eternal security)}

Romans 6:23  For the wages of sin is death; but the gift of God is eternal life through Jesus Christ our Lord.


\textbf{Questions}

  1. Do you believe that you have sinned?

  2. Do you believe that Jesus Christ died on the cross for your sins and rose again from the dead?

  3. If you were to ask Jesus to save you right now would he do it?

  4. How long would you be saved for?

\textbf{If understood}

{\em "Dear Jesus, I know I am a sinner, and I know I deserve to go to hell.  But I
believe that you died on the cross for my sins. Please save me and take me to
heaven when I die. I'm trusting only you, Jesus. Amen." }

Romans 10:13 For whosoever shall call upon the name of the Lord shall be saved.

John 5:24 Verily, verily, I say unto you, He that heareth my word, and
believeth on him that sent me, hath everlasting life, and shall not come into
condemnation; but is passed from death unto life.


    \end{english}%
}
\gdef\ppGospelAF{%
    \begin{afrikaans}%
        \mygospelfont%
        \textsc{Ewige lewe}

\textbf{1. Elkeen is ‘n sondaar}

Rom. 3:10: soos geskrywe is: Daar is niemand regverdig nie, selfs nie een nie.

Rom. 3:23: want almal het gesondig en dit ontbreek hulle aan die heerlikheid van God.

\textbf{2. Elkeen verdien die hel}

Rom. 6:23: Want die loon van die sonde is die dood, maar die genadegawe van God is die ewige lewe in Christus Jesus, onse Here.

Openb. 20:14-15: En die dood en die doderyk is in die poel van vuur gewerp. Dit is die tweede dood.  En as dit bevind is dat iemand nie opgeskryf was in die boek van die lewe nie, is hy in die poel van vuur gewerp.

Openb. 21:8: Maar wat die vreesagtiges aangaan en die ongelowiges en die gruwelikes en moordenaars en hoereerders en towenaars en afgodedienaars en al die leuenaars – hulle deel is in die poel wat brand met vuur en swawel: dit is die tweede dood.
       
\textbf{3. Christus het vir ons sondes gesterf volgens die Skrifte; Hy is begrawe, en Hy is op die derde dag opgewek volgens die Skrifte;}

Rom. 5:8: maar God bewys sy liefde tot ons daarin dat Christus vir ons gesterf het toe ons nog sondaars was.

1 Pet. 2:24: wat self ons sondes in sy liggaam op die kruishout gedra het, sodat ons die sondes kan afsterwe en vir die geregtigheid lewe; deur wie se wonde julle genees is.

\textbf{4. Jy moet in Jesus Christus glo as jou enigste weg na die hemel}

Hand. 16:31: Toe sê hulle: Glo in die Here Jesus Christus en jy sal gered word, jy en jou huisgesin.

Joh. 3:16: Want so lief het God die wêreld gehad, dat Hy sy eniggebore Seun gegee het, sodat elkeen wat in Hom glo, nie verlore mag gaan nie, maar die ewige lewe kan hê.

\textbf{5. Wanneer jy in Jesus Christus glo, gee God jou die ewige lewe (Gelykenis van ‘n geskenk om ewigheids­sekerheid te verduidelik)}

Rom. 6:23: Want die loon van die sonde is die dood, maar die genadegawe van God is die ewige lewe in Chrisus Jesus, onse Here.

Vrae

  1. Glo jy dat jy gesondig het?

  2. Glo jy dat Jesus Christus op die kruis vir jou sondes gesterf het, en uit die dood opgestaan het?

  3. As jy Jesus nou sou vra om jou te red, sal hy dit doen?

  4. Hoe lank sal jy gered bly?


\textbf{Indien verstaan}

{\em “Here Jesus, ek weet ek is ‘n sondaar, en ek verdien om hel to te gaan.  Maar ek glo dat U op die kruis vir my sondes gesterf het.  Red my asseblief, en neem my hemel toe as ek tot sterwe kom.  Ek vertrou alleenlik op jou, Jesus.  Amen.” }

Rom. 10:13 Want: Elkeen wat die Naam van die Here aanroep, sal gered word.

%
    \end{afrikaans}
}
\gdef\ppgospeloutlinepage{%
    \newpage
    %\thispagestyle{noheadings}%
    {% preface
        \baselineskip=7.4pt%
        \begin{paracol}{2}%
            \begin{leftcolumn}\ppGospelEN\end{leftcolumn}%
            \begin{rightcolumn}\ppGospelAF\end{rightcolumn}%
        \end{paracol}%
    }%
}

%%% PAGE 3: TITLE PAGE %%%%%%%%%%%%%%%
% Kerning adjustment: big deal page-count affecter
% Big deal don't hypenate rule: we would save pages by hypenating, but it's ugly
\def\pptitlepage{
\thispagestyle{noheadings}%
%\par\noindent
\null
\vfill
\null\par
{\noindent\hfil\sc KJV 1953 Afrikaans Parallel Bible New Testament 2024\hfil}
\null\par
\null\par
\null
\begin{paracol}{2}%
    \begin{leftcolumn}%
        \begin{english}%
        {\centering
            {\noindent\bookheadingstrut{3\baselineskip}{\mybooktitlefont\mybooktitlesettingsz\mybooktitletextbf{HOLY BIBLE}}}\par% King James Bible
            {\noindent Authorised Version, 1611}\par
            {\noindent New Testament}\par% kjvtex compactt
        }%
        \null\par
        \ifdefined\PPtocEN{\PPtocEN}\fi\par
        \bookheadingstrut{3\baselineskip}\null\par
        {\centering
            {\noindent \em How beautiful are the feet of them that preach the gospel of peace, and bring glad tidings of good things!}\par
        }%
        \end{english}%
    \end{leftcolumn}%
    \begin{rightcolumn}%
        \begin{afrikaans}%
        {\centering
            {\noindent\bookheadingstrut{3\baselineskip}{\mybooktitlefont\mybooktitlesettingsz\mybooktitletextbf{DIE BYBEL}}}\par% King James Bible
            {\noindent Ou Vertaling, 1953}\par
            {\noindent Nuwe Testament}\par% kjvtex compactt
        }%
        \null\par
        \ifdefined\PPtocAF{\PPtocAF}\fi\par
        \bookheadingstrut{3\baselineskip}\null\par
        {\centering
            {\noindent \em Hoe lieflik is die voete van die wat die evangelie van vrede verkondig, van die wat die evangelie van die goeie verkondig!}\par
        }%
        \bookheadingstrut{3\baselineskip}\null\par
        \end{afrikaans}%
    \end{rightcolumn}%
\end{paracol}%
%%%%%%%%%%%%%%%%%%%%%%%%%%%
\null
\vfill
\begin{center}
    \small http://soulwinning.africa/af
\end{center}
%%%%%%%%%%%%%%%%%%%%%
% Commands that write the table of content later
%%%%%%%%%%%%%
\newpage
}
%%% PAGE 4: BLANK PAGE %%%%%%%%%%%%%%%
\def\ppblankagain{%
    \shipout\null\stepcounter{page}%this page intentionally left blank
}
%%%%
\setlength{\columnsep}{24pt}
\setlength{\columnseprule}{0.1pt}%
\ppFrontMatter
\setlength{\columnsep}{12pt}
\setlength{\columnseprule}{0.4pt}%
%\ppblankagain%blank again
%%% PAGE 5: FIRST PAGE OF CONTENT %%%%%%%%%%%%%%%
\newwrite\mytoc\immediate\openout\mytoc=\jobname.section%
\pagestyle{fancy}
\mydocumentsettings % Take II: apply document settings again, in case we changed things
%%% BODY CONTENT %%%
\include{parallel.tex}% Load main content, from parallel.tex
%% Write out stats and stuff for laters
\newwrite\mypages
\immediate\openout\mypages=\jobname.pages%
\addtocounter{page}{-1}% adjust for presentation
\immediate\write\mypages{\unexpanded{\gdef\myshowpageinfo}{\thepage\ pages }}% FIXME: adds one to the page .. weird
\addtocounter{page}{1}% un-adjust for presentation
% FIXME: this is dumb:
% I think we're done here ...
%% Oh wait, we might write out something more here
\myrearmatter% print the rear matter pages (not counted above!)
%%% PAGE n: RECORD PAGE COUNT AND QUIT %%%%
\addtocounter{page}{-1}% adjust for presentatoin
\immediate\write\mypages{\unexpanded{\gdef\mypagecount}{\thepage}}% FIXME: adds one to the page .. weird
\addtocounter{page}{1}% un-adjust for presentation
\immediate\closeout\mytoc%
\immediate\closeout\mypages%
\PPenddocument
%%\bigskip%%\ldots
\end{document}

